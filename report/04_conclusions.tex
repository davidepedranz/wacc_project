\section{Conclusions}
\label{sec:conclusions}

In this project, we have designed, implemented, deploy and tested a scalable and fault-tolerant web application to monitor a Docker Swarm cluster.
We used NoSQL high performant databases to store the data, Kafka as a message queue to support the data streaming and the Play Framework to write totally asynchronous code for the backend.
External load balancing and replication of data and services allows to both scale to high traffic loads and to resist to failures.

\paragraph{Docker Swarm issues and implementation design}
At first we faced some problems when interacting with Docker, since it exposes its APIs as a UNIX socket and Play WS does not support it.
Hence, we decided to expose that socket as a TCP port internally to the cluster with the $socat$ tool, using the following command: "socat -d  -d TCP-L:2375 fork UNIX:/var/run/docker.sock".

Another problem was the first interaction with the Docker APIs.
The Docker daemon exposes both plain Docker and Docker Swarm specific APIs together.
For example, getting the list of running containers of available images is a plain Docker functionality and works at the node level (not at the cluster level).
Thus, we decided to use only Docker Swarm concepts, in particular we concentrated on the services.
Our final implementation allows to list, create, update and delete services in the Docker Swarm cluster.

\paragraph{Future Works}
The application works nicely, but it is not a complete Docker Swarm dashboard.
Future works may add additional functionalities such as visualizing all running containers for each server, opening a shell to a given container to perform some manual operation, visualizing the logs of a service, etc.

From the point of view of existing functionalities, there are some possible improvements.
Password in MongoDB are stored in clear text, and this is obviously not optimal since the developers should never be able to view user's password.
An easy fix for this is to use \texttt{bcrypt}\footnote{\url{https://en.wikipedia.org/wiki/Bcrypt}} to hash the password and store the hash instead.

Though the most important functionalities of the backend are covered with unit tests, we do not have a suite of tests that covers all the code.
Future developments shall also focus on writing more test cases. Additionally, it would be nice to write some end-2-end test to verify the correct interactions between frontend and backend.

The last point is to drop the Docker Swarm Proxy, since it adds an extra point of failure.
This would require the use of a different library with support for UNIX sockets, which make HTTP calls to the Docker Swarm APIs.
